\documentclass[12pt,a4paper]{article}

\usepackage[T2A]{fontenc}
\usepackage[utf8]{inputenc}
\usepackage[russian]{babel}
\usepackage{amsmath,amsfonts,amssymb}
\usepackage{mathrsfs}
\usepackage[margin=2.5cm]{geometry}
\usepackage{setspace}
\usepackage{indentfirst}
\usepackage{graphicx}
\usepackage{hyperref}

% Настройки гиперссылок
\hypersetup{
    colorlinks=true,
    linkcolor=blue,
    citecolor=blue,
    urlcolor=blue
}

% Настройки отступов
\setlength{\parindent}{1.25cm}
\onehalfspacing

\title{\textbf{Проективно-модальная онтология образа В.С. Соловьёва в критической статье Д.С. Мережковского «Немой пророк»: опыт применения формального аппарата к литературно-критическому дискурсу}}

\author{Е.Г. Луговская, Е.К. Грудина}

\date{}

\begin{document}

\maketitle

\begin{abstract}
\noindent В статье предложена проективно-модальная реконструкция образа В.С. Соловьёва в критической статье «Немой пророк» из сборника Д.С. Мережковского «В тихом омуте» на основе аппарата Проективно-модальной онтологии (ПМО) В.И. Моисеева. Критический метод Мережковского формализован как систематическое варьирование модуса личности философа на множестве темпоральных, онтологических и эпистемологических моделей с выявлением собственных и несобственных моделей. Установлено, что базовый модус Соловьёва идентифицируется Мережковским не как нейтральный субъект, но как имманентно гностический (созерцательный), что делает революционный прагматизм несобственной моделью, на которой проектор не может образовать моду. Трагедия «немого пророка» интерпретируется как онтологическая апория, возникающая из структурной невозможности образования аутентичной моды на модели исторического действия при одновременной невыразимости истинной (профетической) моды в публичном дискурсе.

\vspace{0.5em}
\noindent\textbf{Ключевые слова:} проективно-модальная онтология, В.С. Соловьёв, Д.С. Мережковский, экзистенциальное портретирование, модус-мода-модель, гностицизм-прагматизм, темпоральность, трагедия немоты, формализация критического метода, ПМО в литературоведении
\end{abstract}

\section{К вопросу о формализации философского портретирования}

Владимир Сергеевич Соловьёв (1853--1900) в метадискурсе русской культуры функционирует как фигура парадоксальная: образ «великого учителя» символистов (Блок, 1906/1980; Белый, 1907/1994) сосуществует с маркировкой его «двойственности» и «нераскрытости» (Розанов, 1900; Трубецкой, 1913). Критическая рецепция Д.С. Мережковского в статье «Немой пророк» (1908) представляет уникальную попытку экзистенциального портретирования через операционализацию бинарной оппозиции «созерцание/деяние», актуализированной в общественном дискурсе начала XX века.

Риторико-герменевтическое исследование материала статьи позволило выявить лингвориторическую структуру речевого портретирования образа Вл. Соловьева через ассоциативно-семантические поля (АСП) концептов «философ-созерцатель» и «человек-деятель», концепты «двойственности» и «скрытости», используемые автором в качестве риторического механизма создания семантического напряжения.

Настоящее исследование направлено на применение аппарата Проективно-модальной онтологии (ПМО) В.И. Моисеева (Моисеев, 2002, 2004) к анализу образа Соловьёва у Мережковского. ПМО как метаязык описания экстраполирована на литературно-критический дискурс Мережковского и позволяет реконструировать концептуальную структуру образа В.С. Соловьёва через онтологическую архитектонику трагедии «немого пророка».

\section{Введение в Проективно-модальную онтологию}

Согласно В.И. Моисееву (2002, 2004), Проективно-модальная онтология основана на семиместном предикате:

\begin{equation}
\mathrm{Mod}(a,b,c,f,d,h,\alpha)
\end{equation}

где:
\begin{itemize}
\item $a$ -- \textbf{мода} (аспект, проявление)
\item $b$ -- \textbf{модус} (источник, генератор бытия)
\item $c$ -- \textbf{модель} (ограничивающее условие)
\item $f$ -- \textbf{проектор} (операция ограничения модуса до моды)
\item $d$ -- \textbf{модуль} (начало расширения)
\item $h$ -- \textbf{сюръектор} (операция расширения моды до модуса)
\item $\alpha$ -- \textbf{спецификатор} (контекст определения)
\end{itemize}

Читается: «В контексте $\alpha$ $a$ есть мода модуса $b$ в модели $c$ с проектором $f$, и $b$ есть модус моды $a$ в модуле $d$ с сюръектором $h$».

Базовая нотация: $X = Y\acute{Z}$ -- «$X$ есть $Y$-при-условии-$Z$», где проектор $\acute{\pi}(Y,Z) = X$ ограничивает модус до моды.

Критически важно понятие \textbf{собственных моделей} модуса $M(Y)$ (Моисеев, 2004: 218--219): не на всех условиях (моделях) модус способен образовывать свои моды. Для модуса $Y$ модель $Z$ является \textbf{несобственной}, если:

\begin{equation}
Z \notin M(Y) \Rightarrow \acute{\pi}(Y,Z) = \emptyset
\end{equation}

Проектор не определён, мода не образуется. Это не выбор, но \textbf{онтологическая невозможность}.

\section{ПМО-структура образа Соловьёва в критике Мережковского}

\subsection{Базовый модус: гностик-созерцатель}

«Вл. Соловьев -- гностик, может быть, \textbf{последний великий гностик всего христианства}». (Мережковский, 1914: 133)

Мережковский определяет Соловьёва не как нейтральный субъект, способный быть либо созерцателем, либо деятелем, но как \textbf{имманентно гностического} (созерцательного) модуса:

«Для него сущность догмата открывается не воле сначала и потом разуму, а, наоборот, \textbf{сначала разуму, потом воле}. Он -- рационалист, как всякий гностик». (Там же)

\textbf{Формализация базового модуса:}

\begin{equation}
Y_{\text{Соловьёв}} = \text{Гностик-созерцатель}
\end{equation}

Структурные предикаты модуса:
\begin{itemize}
\item $\mathrm{Pr}_1(Y)$: Приоритет разума (гнозис) над волей
\item $\mathrm{Pr}_2(Y)$: «Богоделание вытекает из богопознания»
\item $\mathrm{Pr}_3(Y)$: Рационалистичность
\item $\mathrm{Pr}_4(Y)$: Консервативность
\item $\mathrm{Pr}_5(Y)$: Реставраторство
\end{itemize}

\subsection{Собственные модели: темпоральная триада}

Мережковский выявляет три \textbf{собственные модели} $M(Y_{\text{Соловьёв}})$, на которых модус образует свои аутентичные моды:

\subsubsection{Модель 1: Прошлое}

\begin{equation}
X_{\text{реставратор}} = Y\acute{Z}_{\text{прошл}}
\end{equation}

Текстовые маркеры:
\begin{itemize}
\item «\textbf{Розовый башмачок} -- безнадежная романтика прошлого» (С. 131)
\item «Былое надежно; будущее страшно» (С. 132)
\item «Лучи \textbf{заходящего солнца}, лампадный свет \textbf{вечерний}...» (С. 132)
\end{itemize}

Мода: $X_{\text{реставратор}} = $ «тайный славянофил», романтик былого

\subsubsection{Модель 2: Настоящее}

\begin{equation}
X_{\text{консерватор}} = Y\acute{Z}_{\text{наст}}
\end{equation}

Текстовые маркеры:
\begin{itemize}
\item «Не разрушать и не созидать, а \textbf{сохранять и поддерживать, подпирать валящееся здание}» (С. 131)
\item «Остановить, запрудить всемирный поток разрушения» (С. 132)
\end{itemize}

Мода: $X_{\text{консерватор}} = $ «подпирающий валящееся», консервирующий

\subsubsection{Модель 3: Будущее как эсхатон}

\begin{equation}
X_{\text{эсхатолог}} = Y\acute{Z}_{\text{эсх}}
\end{equation}

Текстовые маркеры:
\begin{itemize}
\item «Страх будущего -- "антихристов страх"» (С. 132)
\item «Последняя и единственная революция для него -- переворот уже не исторический, а космический -- кончина мира» (С. 132)
\end{itemize}

Мода: $X_{\text{эсхатолог}} = $ «пророк антихриста», апокалиптик

\subsection{Несобственная модель: революционный прагматизм}

Для модуса Соловьева мода революционера не образуется:

\begin{equation}
Z_{\text{прагм}} \notin M(Y_{\text{Соловьёв}}) \Rightarrow \acute{\pi}(Y, Z_{\text{прагм}}) = \emptyset
\end{equation}

Текстовые обоснования:
\begin{itemize}
\item «Стихия революционная \textbf{чужда} ему \textbf{навеки и безнадежно}» (С. 132)
\item «\textbf{Не только революция, но и реформация, не могли бы вспыхнуть} от соловьевского гнозиса» (С. 133)
\end{itemize}

\section{Трагедия «немоты» как онтологическая апория}

«Немой пророк» -- оксюморонное сочетание, содержащее структурный код трагедии:
\begin{itemize}
\item \textbf{Пророк} = обладатель профетического знания (модус)
\item \textbf{Немой} = неспособность к артикуляции (блокировка проектора)
\end{itemize}

\textbf{Формализация трагедии:}

\begin{equation}
\text{«Немой пророк»} = \lim_{X \to X_{\text{истина}}} \frac{\text{Выразимость}(X)}{\text{Истинность}(X)} = 0
\end{equation}

Чем ближе к профетической истине, тем меньше возможность выражения. Предел -- абсолютная немота при абсолютной истинности.

\subsection{Метафора «омута» как предельный переход}

Омут понимается как предельный переход:

\begin{equation}
\lim_{t \to \infty} [Y\acute{Z}_{\text{гнозис}}(t) \oplus Y\acute{Z}_{\text{прагм}}(t)] = \Omega
\end{equation}

где $\Omega$ -- «омут», $\oplus$ -- оператор слияния (модусная сумма).

Но:
\begin{equation}
\forall t_{\text{конечное}}: Y\acute{Z}_{\text{гнозис}}(t) \cap Y\acute{Z}_{\text{прагм}}(t) = \emptyset
\end{equation}

Слияние отложено до эсхатона. В исторической реальности (конечное $t$) слияния нет. Соловьёв существует \textbf{до} омута, умирая «как безумец» до достижения синтеза.

\section{Выводы}

Применение аппарата Проективно-модальной онтологии В.И. Моисеева к анализу образа В.С. Соловьёва в критической статье Д.С. Мережковского «Немой пророк» позволило:

\begin{enumerate}
\item \textbf{Формализовать базовый модус} не как нейтральный субъект, но как имманентно гностический (созерцательный), что делает революционный прагматизм \textbf{несобственной моделью}.

\item \textbf{Выявить темпоральную архитектонику} образа через триаду собственных моделей: прошлое (реставраторство), настоящее (консервация), эсхатологическое будущее (страх антихриста).

\item \textbf{Скорректировать понимание «омута»} не как актуального синтеза, но как \textbf{эсхатологического предельного перехода}, недостижимого в конечном времени.

\item \textbf{Формализовать трагедию «немоты»} как онтологическую апорию: $X_{\text{истинное}} \cap X_{\text{выраженное}} = \emptyset$, где профетическая истина структурно невыразима в публичном дискурсе.

\item \textbf{Диагностировать три типа онтологических апорий}: несовпадение модуса и эпохи, блокировка проектора на несобственной модели, расщепление аутентичной и неаутентичной мод.
\end{enumerate}

Экзистенциальное портретирование Мережковского реконструировано как систематическая процедура проективного варьирования, изоморфная методу античной диалектики. Предложен формальный метаязык для описания концептуальных структур, выявленных лингвориторическим анализом.

\section*{Благодарности}

Статья подготовлена в рамках деятельности Лаборатории Философии Слова Интегрального Сообщества.

\begin{thebibliography}{99}

\bibitem{belyj1994}
Белый А. Воспоминания о Блоке. \textit{Александр Блок в воспоминаниях современников}. М.: Художественная литература, 1994. (Оригинальная работа опубликована в 1907).

\bibitem{blok1980}
Блок А. Рыцарь-монах. \textit{Собрание сочинений в 8 томах}. Т. 5. М.: Художественная литература, 1980. (Оригинальная работа опубликована в 1906).

\bibitem{merezhkovskij1914}
Мережковский Д.С. Немой пророк. \textit{Полное собрание сочинений}. Т. 16. М.: Издание И.Д. Сытина, 1914. С. 128--135. (Оригинальная работа опубликована в 1908).

\bibitem{moiseev2002}
Моисеев В.И. Логика всеединства. М., 2002.

\bibitem{moiseev2004}
Моисеев В.И. Проективно-модальная онтология и некоторые её приложения. \textit{Логические исследования}. Вып. 11. М.: Наука, 2004. С. 215--229. \url{https://iphras.ru/uplfile/logic/log11/Li_11_Moiseev.pdf}

\end{thebibliography}

\end{document}